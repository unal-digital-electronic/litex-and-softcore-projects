\documentclass{beamer}

% \usepackage{lipsum}             % for dummy text
% \usepackage{stix}               % use the STIX font (of course you can delete this line)
\usepackage[utf8]{inputenc}

% \usetheme{Notebook}
% \usetheme{Blackboard}
\usetheme{DarkConsole}
% \usetheme{metropolis}
% \usepackage{textcomp}

% Insert all in preambule
\usepackage{listings}
\usepackage{xcolor}

\definecolor{codegreen}{rgb}{0,0.6,0}
\definecolor{codegray}{rgb}{0.5,0.5,0.5}
\definecolor{codepurple}{rgb}{0.58,0,0.82}
\definecolor{backcolour}{rgb}{0.95,0.95,0.92}

\lstdefinestyle{mystyle}{
  % backgroundcolor=\color{backcolour},
  commentstyle=\color{codegreen},
  keywordstyle=\color{magenta},
  numberstyle=\tiny\color{codegray},
  stringstyle=\color{codepurple},
  % basicstyle=\ttfamily\footnotesize,
  % breakatwhitespace=false,
  % breaklines=true,
  captionpos=b,
  % keepspaces=true,
  % numbers=left,
  % numbersep=5pt,
  showspaces=false,
  showstringspaces=false,
  showtabs=false,
  tabsize=2
}

\lstset{style=mystyle}


\title{Título \texttt{x}}
\subtitle{Subtítulo}
\author{Author\footnote{\texttt{e-mail}\\}}

\begin{document}

\begin{frame}
  \maketitle
\end{frame}

\begin{frame}{Contenido}
  \tableofcontents
\end{frame}

\section{Variables de entorno de Zephyr}

\begin{frame}[fragile]{Exportando variables de entorno}
  Variables de entorno usadas por \textit{zephyr}
% \usepackage{listings}
\begin{lstlisting}[language=Bash, tabsize=2, label=_, basicstyle=\small]
export ZEPHYR_TOOLCHAIN_VARIANT=zephyr
export ZEPHYR_SDK_INSTALL_DIR=~/zephyr-sdk-0.11.3
export ZEPHYR_BASE=~/zephyrproject/zephyr
\end{lstlisting}

Variables entorno para usar un toolchain alternativo

\begin{lstlisting}[language=Bash, tabsize=2, label=code_, basicstyle=\footnotesize]
export ZEPHYR_TOOLCHAIN_VARIANT=cross-compile
export CROSS_COMPILE=~/miniconda3/envs/fpga/bin/riscv32-elf-
\end{lstlisting}
\end{frame}

\section{Configuración y compilación con zephyr}

\begin{frame}[fragile]{Usando West y Ninja}
  % \usepackage{listings}
  \begin{lstlisting}
west build -p auto -b board samples/hello_world
cmake -B build -GNinja -DBOARD=board samples/hello_world
  \end{lstlisting}
% \usepackage{listings}
\begin{lstlisting}[language=Bash, tabsize=2, label=code_1, basicstyle=\small]
cd build/
ninja run
ninja clean
ninja 
\end{lstlisting}

\end{frame}

\begin{frame}[fragile]{Usando Make}
% \usepackage{listings}
\begin{lstlisting}[language=Bash, tabsize=2, label=code_, basicstyle=\small]
cmake -B build -DBOARD=board samples/hello_world
cd build/
make
\end{lstlisting}
\end{frame}

\section{zephyr y dbg}

\begin{frame}[fragile]{Haciendo uso de dbg}
% \usepackage{listings}
\begin{lstlisting}[language=Bash, tabsize=2, label=code_, basicstyle=\small]
ninja debugserver
\end{lstlisting}

\begin{lstlisting}[upquote=true, language=Bash, tabsize=2, label=code_, basicstyle=\small, columns=fullflexible, showspaces=false, showstringspaces=false]
gdbgui --gdb \
  ~/zephyr-sdk-0.11.3/riscv64-zephyr-elf/bin/riscv64-zephyr-elf-gdb\
  --gdb-args=' -ex "target remote localhost:1234" ./zephyr.elf'
\end{lstlisting}

\end{frame}


\end{document}
